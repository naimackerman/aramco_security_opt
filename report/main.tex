% =================================================================
% INCOSE Conference LaTeX Template V1.2 (Release Date: November 4th, 2025)
% Copyright (c) 2025 INCOSE
% 
% This template is provided for use in preparing manuscripts for
% INCOSE conferences. You may use, modify, and 
% distribute this template for academic and professional purposes.
% 
% This template is provided "as is" without warranty of any kind.
% The author(s) disclaim all warranties, express or implied,
% including but not limited to warranties of merchantability and
% fitness for a particular purpose.

% =================================================================

\documentclass[11pt,letterpaper]{article} % Remove this line if using A4 size.
%\documentclass[11pt,a4]{article} % A4 is also accepted and is typically used for non-US submissions.

% ---------- Core layout ----------
\usepackage[
  letterpaper,
  left=0.6in,right=0.6in,top=0.6in,bottom=0.6in,
  headheight=85pt,headsep=-45pt
]{geometry}
\raggedbottom
\usepackage{graphicx}
\graphicspath{{./}{figures/}}
\usepackage{float}
\usepackage{amsmath}
\usepackage{tikz}
% \usepackage[letterpaper,top=2cm,bottom=2cm,left=3cm,right=3cm,marginparwidth=1.75cm]{geometry}
% Useful packages
\usepackage{amssymb}
% \usepackage[colorlinks=true, allcolors=blue]{hyperref}


% ---------- Captions & float spacing ----------
\usepackage{caption}
\captionsetup{
  font={sf,bf,footnotesize},
  labelfont={sf,bf,footnotesize},
  justification=centering,
  labelsep=period,
  hypcap=false
}
\setlength{\textfloatsep}{8pt}
\setlength{\floatsep}{6pt}
\setlength{\intextsep}{8pt}
\setlength{\abovecaptionskip}{12pt}
\setlength{\belowcaptionskip}{1pt}

% ---------- Tables / lists ----------
\usepackage{booktabs}
\usepackage{array}
\usepackage{tabularx}
\usepackage{enumitem}
\setlist{nosep}
\usepackage[table]{xcolor}
\definecolor{tableheader}{HTML}{D9D9D9}
\newcolumntype{P}[1]{>{\sffamily\centering\arraybackslash}p{#1}}
\newcolumntype{Y}{>{\sffamily\centering\arraybackslash}X}
\newcolumntype{A}[1]{>{\raggedright\arraybackslash}p{#1}}
\newcolumntype{M}[1]{>{\raggedright\arraybackslash}m{#1}}

% ---------- Fonts ----------
\usepackage[T1]{fontenc}
\usepackage[utf8]{inputenc}
\usepackage{PTSerif}
\usepackage[scaled]{helvet}
\renewcommand{\sfdefault}{phv}
\newcommand{\headingfont}{\sffamily}

% ---------- Headings ----------
\usepackage{titlesec}
\setcounter{secnumdepth}{0}

% Heading 1
\titleformat{\section}
  {\headingfont\bfseries\raggedright\fontsize{18pt}{18pt}\selectfont}{}{0.75em}{}
% Heading 2
\titleformat{\subsection}
  {\headingfont\bfseries\raggedright\fontsize{15pt}{16pt}\selectfont}{}{0.75em}{}
% Heading 3
\titleformat{\subsubsection}
  {\headingfont\bfseries\raggedright\fontsize{12pt}{14pt}\selectfont}{}{0.75em}{}

% Heading spacing
\titlespacing*{\section}{0pt}{7pt}{6pt}
\titlespacing*{\subsection}{0pt}{7pt}{6pt}
\titlespacing*{\subsubsection}{0pt}{7pt}{6pt}

\newcommand{\miniheading}[1]{%
  \par\noindent{\headingfont\bfseries\fontsize{12pt}{14pt}\selectfont #1}\par\vspace{4pt}%
}

% ---------- Paragraphing ----------
\setlength{\parindent}{0pt}
\setlength{\parskip}{6pt plus 1pt minus 1pt}

% ---------- Page numbers ----------
\usepackage{fancyhdr}
\pagestyle{fancy}
\fancyhf{}
\fancyhfoffset[R]{18pt}
\setlength{\footskip}{18pt}
\fancyfoot[R]{\sffamily\bfseries\footnotesize \thepage}
\renewcommand{\headrulewidth}{0pt}
\renewcommand{\footrulewidth}{0pt}

% INCOSE logo
\fancypagestyle{firstpage}{
  \fancyhf{}
  \fancyhfoffset[R]{18pt}
  \fancyhead[R]{%
    \smash{\raisebox{0pt}[0pt][0pt]{
      \begingroup\setlength{\fboxsep}{20pt}
        \colorbox{white}{\includegraphics[height=0.6in]{template-images/incose-logo.jpg}}%
      \endgroup
    }}
  }
  \fancyfoot[R]{\sffamily\bfseries\footnotesize \thepage}
  \renewcommand{\headrulewidth}{0pt}
  \renewcommand{\footrulewidth}{0pt}
}
% ---------- Title Formatting ----------
\makeatletter
\@ifundefined{theauthor}{}{\let\theauthor\relax}
\makeatother
\usepackage{titling}
\pretitle{\headingfont\bfseries\fontsize{24pt}{26pt}\selectfont\raggedright}
\posttitle{\par\vspace{-.3in}}
\preauthor{}\postauthor{}
\author{\mbox{}}
\date{}
\setlength{\droptitle}{-3.2\baselineskip}

% ---------- Author cards ----------
\newcommand{\authorcard}[5]{%
  {\headingfont\bfseries\fontsize{12pt}{14pt}\selectfont #1}\par
  {\headingfont\bfseries\fontsize{12pt}{14pt}\selectfont #2}\par
  {\headingfont\bfseries\fontsize{12pt}{14pt}\selectfont #3}\par
  {\headingfont\bfseries\fontsize{12pt}{14pt}\selectfont #4}\par
  {\headingfont\bfseries\fontsize{12pt}{14pt}\selectfont #5}\par
}

% ---------- Biography photo placeholder and entry ----------

\makeatletter
\newcommand{\authorpic}[1]{%
    \includegraphics[width=0.6in,height=0.6in,keepaspectratio,clip]{#1}%
}
\makeatother

\newcommand{\authorbioentry}[3]{%
  \noindent\begin{tabular}{@{}m{0.5in} M{\dimexpr\columnwidth-0.5in\relax}@{}}
    \authorpic{#1} & \textbf{#2}\par #3
  \end{tabular}\par\medskip
}

% ---------- Safe figure include ----------
\makeatletter
\newcommand{\colfig}[2][]{%
  \IfFileExists{#2}{\includegraphics[width=\linewidth,#1]{#2}}{%
    \fbox{\parbox[b][1.5in][c]{\linewidth}{\centering \textit{Missing figure: }#2}}}%
}
\makeatother

% ---------- References: APA via biblatex/biber ----------
\let\theauthor\relax
\usepackage{csquotes}
\usepackage[style=apa,backend=biber]{biblatex}
\addbibresource{references.bib}

% ---------- Highlight callouts ----------
\usepackage{changepage}
\newenvironment{highlight}[1][0.25in]{%
  \begin{adjustwidth}{#1}{#1}\itshape}{\end{adjustwidth}}

% ---------- Two-column setup ----------
\usepackage{multicol}
\setlength{\columnsep}{18pt}

% ---------- Hyperlinks ----------
\usepackage[hidelinks]{hyperref}

% =========================
% ===== Title & Authors ===
% =========================
\title{Optimized Human-Robot Co-Dispatch \\ Planning for Petro-Site Surveillance \\ Systems under Varying Criticalities}

\begin{document}
\maketitle
\thispagestyle{firstpage}

% ---- Authors ----
% ---- For the initial paper submission, do not include any author information. For the final paper submission, format author information as shown below. ------------------

\noindent
\begin{tabular*}{\textwidth}{@{\extracolsep{\fill}} A{0.32\textwidth} A{0.32\textwidth} A{0.32\textwidth}}
  \authorcard{Nur Ahmad Khatim}{Institute Technology of Sepuluh Nopember}{Surabaya, Indonesia}{}{nurahmadkhatim@gmail.com} &
  % \authorcard{Author Two}{Organization}{Street Address}{City, Province, Postal}{author.two@email.com} &
  % \authorcard{Author Three}{Organization}{Street Address}{City, Province, Postal}{author.three@email.com} \\
  % \multicolumn{3}{@{}c@{}}{\rule{0pt}{0.9\baselineskip}} \\[-0.2\baselineskip]
  % \authorcard{Author Four}{Organization}{Street Address}{City, Province, Postal}{author.four@email.com} &
  % \authorcard{Author Five}{Organization}{Street Address}{City, Province, Postal}{author.five@email.com} &
  \authorcard{Yan Akhra Pratama}{Institute Technology of Sepuluh Nopember}{Surabaya, Indonesia}{}{pratamaakhra@gmail.com} &
 \authorcard{Mansur M. Arief}{King Fahd University of Petroleum and Minerals}{Dhahran, Saudi Arabia}{}{mansur.arief@kfupm.edu.sa} \\
\end{tabular*}
% \addvspace{.75in}

% ---- Two columns begin immediately after authors ----
\begin{multicols*}{2}
\raggedcolumns

% ---- Copyright ----
{\headingfont\bfseries\fontsize{8pt}{12pt}\selectfont
Copyright~\textcopyright~ \the\year{} by the author(s). Permission granted to INCOSE to publish and use.}
\\
% =========================
% ===== Abstract/Keywords =
% =========================
\phantomsection
\miniheading{Abstract}
This document presents instructions for preparing a manuscript for INCOSE conferences. Please note that your submission will not appear in proceedings unless it conforms exactly to the required format. Please proofread your submissions carefully for typographical, spelling, or grammatical errors. We strongly encourage you to use this document as a template for developing your own manuscript.

\phantomsection
\subsubsection{Keywords}
Provide a few keywords, separated by commas.

% =========================
% ===== Main Content ======
% =========================
\section{Introduction}

The global petroleum infrastructure represents one of the most extensive and geographically dispersed critical infrastructure networks, with over 2.7 million miles of oil and gas pipelines traversing the United States alone, and similar dense networks spanning the Arabian Peninsula, the North Sea, and other major hydrocarbon basins (\cite{coburn2020oil, dietl2021global, klass2014transporting}). These networks exhibit a hierarchical topology characterized by high-criticality nodes (refineries, processing facilities, storage terminals, and compressor stations) interconnected by extensive linear assets of varying criticality. The protection of such infrastructure has emerged as a paramount concern for national security, economic stability, and environmental stewardship (\cite{pashchenko2024main, bajpai2007securing}), particularly as threat vectors have evolved to encompass both physical intrusion and sophisticated cyber-physical attacks (\cite{mohammed2022cybersecurity}).

Saudi Aramco, the world's largest integrated energy company managing the planet's most extensive hydrocarbon network, exemplifies the operational complexity inherent in securing petroleum infrastructure at scale (\cite{alsuwailem2022integrated}). The company's Dhahran headquarters and its distributed operational assets span a spectrum of criticality levels, from Tier-1 facilities requiring continuous 24/7 active monitoring to geographically extensive pipeline corridors where periodic patrol suffices. This heterogeneous demand landscape necessitates a correspondingly differentiated security response capability---one that balances the rapidity and precision of autonomous robotic systems with the contextual judgment and ethical decision-making capacity of human operators (\cite{zhang2021safety, al2022managing}).

The advent of Fourth Industrial Revolution (4IR) technologies has catalyzed a paradigm shift in industrial security operations. Unmanned aerial vehicles (UAVs), autonomous ground vehicles (AGVs), and integrated sensor networks now offer capabilities that complement and extend human surveillance capacity. However, the integration of these heterogeneous assets into a coherent systems presents a complex systems engineering challenge. The DARPA challenge showcased how multi-robot systems deployed without being cognizant of their capabilities could crash into one another and lead to mission failures (\cite{agha2021nebula}). The problem is not merely technological; it is fundamentally one of resource allocation, facility location, service level optimization, and human-machine teaming under operational constraints.

Traditionally, the problem of facility services coverage is modeled  as covering problem in the operations research literature. The classical models traditionally deal with minimizing facility costs while covering all the demand points, assuming uncapacitated server agents, or maximizing the demands served given available servers (\cite{li2011covering}). The former problem is coined the set covering problem (\cite{toregas1971location}) while the latter is called maximal covering problem (\cite{megiddo1983maximum}). Extensions to these models have been proposed to cover various service and demand characterizations (see \cite{li2011covering} for a review). The main assumption that we relax in this paper is the incorporation of the deployment of hybrid human-robot security teams across petroleum infrastructure with varying criticality levels. We formulate the problem as a Capacitated Facility Location Problem (CFLP) variant that explicitly incorporates: (i) tiered infrastructure criticality with differentiated service level agreements (SLAs); (ii) human-robot co-dispatch with varying supervision ratios reflecting technological maturity and human-in-the-loop requirements; (iii) set covering constraints ensuring redundant surveillance coverage under common petroleum site settings; and (iv) capacity constraints on command center resources. The objective is to minimize total deployment cost while satisfying all coverage requirements, response time constraints, and human-in-the-loop supervision mandates.

The contributions of this work are fourfold. First, we develop a novel mathematical formulation that bridges the classical facility location literature with the emerging domain of human-robot teaming for critical infrastructure protection. Second, we explicitly model the hierarchical criticality structure of petroleum infrastructure, distinguishing between high-value concentrated assets (refineries, terminals) requiring stringent SLAs and distributed linear assets (pipelines) with relaxed but spatially extensive coverage requirements. Third, we incorporate human supervision constraints that reflect the ethical and regulatory requirements for autonomous system deployment in security-critical applications. Fourth, we demonstrate the model's applicability through a case study of from Dhahran district with dummy data points, providing actionable deployment recommendations across multiple technology maturity scenarios.

% The remainder of this paper is organized as follows. Section 2 reviews the relevant literature spanning critical infrastructure protection, facility location optimization, and human-robot collaboration. Section 3 presents the formal problem description and mathematical formulation. Section 4 details the solution methodology combining exact optimization with scalable heuristics. Section 5 describes the case study data and scenario design. Section 6 presents computational results and sensitivity analyses. Section 7 concludes with implications for practice and directions for future research.

\section{Literature Review}

This paper bridges three research domains: critical infrastructure protection (CIP) for petroleum facilities, facility location problems in operations research, and human-robot collaboration for surveillance tasks.

\subsection{Critical Infrastructure Protection}

Petroleum infrastructure protection has received sustained attention following heightened post-2001 security concerns. The TSA Pipeline Security Guidelines and DHS frameworks classify facilities based on criticality---considering target viability, importance to energy supply, and weaponization potential (\cite{bajpai2007securing, pashchenko2024main}). Unlike discrete facilities, pipeline networks exhibit continuous spatial extent with critical nodes distributed across remote terrain. Church and Scaparra's $r$-interdiction models identify infrastructure elements whose loss maximally degrades system performance, establishing that strategic fortification must account for attacker-defender dynamics (\cite{church2004identifying, scaparra2008bilevel}).

Contemporary petroleum security increasingly leverages autonomous systems. Saudi Aramco has deployed drones for pipeline surveillance and perimeter monitoring, reducing inspection times by up to 90\% at facilities like the Uthmaniyah Gas Plant (\cite{alwalaie2021aramco}). However, the literature lacks optimization frameworks for strategic placement of command centers from which hybrid human-robot teams are dispatched.

\subsection{Facility Location Problems}

The facility location problem (FLP) originates from Weber's industrial location theory and Hakimi's work on optimal switching center placement (\cite{hakimi1964optimum, hakimi1965optimum}). The Set Covering Location Problem (SCLP) seeks minimum facilities ensuring all demand points are within a service threshold (\cite{toregas1971location}), while the Maximal Covering Location Problem (MCLP) maximizes covered demand given fixed facilities (\cite{church1974maximal}). The Capacitated Facility Location Problem (CFLP) adds constraints on demand served per facility (\cite{daskin2013network}).

Extensions include backup coverage concepts (\cite{hogan1986backup}) and bilevel interdiction-fortification models (\cite{scaparra2008bilevel}). The emergency medical services (EMS) literature has developed tiered response systems where different vehicle types serve different call priorities (\cite{farahani2012covering, jayarathna2024covering}), which provides conceptual parallels to human-robot teaming. However, integration of heterogeneous resource types with varying capabilities and supervision requirements remains unaddressed.

\subsection{Human-Robot Collaboration}

Autonomous security robots now offer 24/7 patrolling with human detection and thermal imaging, deployed across critical infrastructure in over 15 countries (\cite{adams2024security}). Despite these capabilities, ethical and regulatory considerations require human oversight for escalation decisions, manifesting as supervision ratio constraints (\cite{chen2011supervisory}).

The human-robot teaming (HRT) literature identifies operator multitasking, trust calibration, and cognitive workload as key factors affecting collaboration (\cite{chen2011supervisory, haring2021understanding}). For petroleum applications, UAVs with YOLO-based detection enable real-time defect identification (\cite{alwalaie2021aramco}). Operations research has addressed human-robot task allocation in warehouse environments (\cite{boysen2019warehouse}), but the strategic facility location problem for security teams remains unexamined.

\subsection{Research Gaps and Contributions}

Three gaps emerge at the intersection of these domains. First, CIP literature lacks optimization models accounting for hybrid human-robot workforces. Second, FLP literature has not incorporated human-robot operational constraints including supervision ratios and differential efficiency rates. Third, HRT literature focuses on operational coordination with limited attention to strategic facility location.

This paper addresses these gaps through an integrated formulation combining: (i) tiered criticality distinguishing high-value concentrated facilities from distributed assets; (ii) capacitated facility location with SLA constraints; (iii) explicit human-robot modeling with differential efficiency and costs; (iv) supervision constraints ensuring ethical oversight; and (v) minimum utilization requirements. We term this the \textit{Human-Robot Co-Dispatch Facility Location Problem} (HRCD-FLP).


% \subsection{Theoretical Foundations}

% Our formulation draws on several theoretical foundations from operations research and systems engineering. The coverage constraints extend the classical SCLP by requiring that each demand site receive sufficient surveillance capacity, measured in Surveillance Coverage Units (SCU), from the resources assigned to its covering facility. This generalizes the binary coverage concept to a continuous capacity requirement.

% The human-robot resource model treats the two resource types as imperfect substitutes characterized by differential efficiency rates and costs. The supervision ratio constraint $z_{\text{Human}} \geq \alpha \cdot z_{\text{Robot}}$ enforces a minimum human presence that increases linearly with the robot fleet size. This constraint reflects both regulatory requirements for human oversight and the practical limitations of supervisory span of control.

% The single-sourcing constraint, requiring each demand site to be served by exactly one command center, simplifies the command and control structure and is common in security applications where unified responsibility is preferred. This constraint transforms the problem from a transportation-type formulation to a location-allocation problem, increasing computational tractability while reflecting operational preferences.

% Finally, the minimum utilization constraint ensures that opened facilities achieve a threshold occupancy rate, preventing the proliferation of underutilized infrastructure. This constraint captures the economies of scale in facility operations and the fixed costs associated with maintaining command center readiness.

% Together, these elements constitute a novel variant of the CFLP that we term the \textit{Human-Robot Co-Dispatch Facility Location Problem} (HRCD-FLP). The following section presents the formal mathematical specification of this problem.


\section{Model}
Saudi Aramco, as the world's largest integrated energy and chemicals company, manages a vast and uniform critical infrastructure network. To secure these assets in alignment with Saudi Vision 2030, the company is transitioning toward smart security operations that leverage autonomous systems. This transition requires a strategic decision-support framework to optimize the deployment of hybrid human-robot security teams.

The problem is formulated as a Multi-Level Capacitated Facility Location Problem (ML-CFLP) with heterogeneous resources. The objective is to determine the optimal locations and operational levels (High, Medium, Low) of security command centers, the assignment of demand sites to these centers, and the specific mix of human and robotic resources required at each facility. This decision must minimize total infrastructure and operational costs while satisfying strict Service Level Agreements (SLAs), capacity constraints, and human-in-the-loop supervision requirements.

\subsection{Problem Description}
We consider a set of candidate locations $I$ and a set of demand sites $J$. Each candidate location can be developed into a command center with a specific operational level $l \in L = \{\text{High}, \text{Medium}, \text{Low}\}$. Each level is characterized by distinct cost structures, resource capacities, and response time capabilities. 

Demand sites require security coverage measured in Surveillance Coverage Units (SCU). This demand is satisfied by a mix of two resource types $k \in K = \{\text{Robot}, \text{Human}\}$. The mix is governed by site-specific characteristics (e.g., complexity requiring more human intervention) and global supervision policies.

\subsection{Mathematical Formulation}

\subsubsection{Sets and Indices}
\begin{itemize}
    \item $I$: Set of candidate command center locations, indexed by $i$.
    \item $J$: Set of demand sites, indexed by $j$.
    \item $L$: Set of facility levels (High, Medium, Low), indexed by $l$.
    \item $K$: Set of resource types (Robot, Human), indexed by $k$.
\end{itemize}

\subsubsection{Parameters}
\begin{itemize}
    \item \textbf{Costs:}
    \begin{itemize}
        \item $F_{il}$: Fixed construction and overhead cost for facility $i$ at level $l$.
        \item $C_{ik}$: Unit deployment cost for resource $k$ at location $i$.
    \end{itemize}
    \item \textbf{Capacities and Capabilities:}
    \begin{itemize}
        \item $MAXCAP_{lk}$: Maximum capacity of resource $k$ for a facility at level $l$.
        \item $MINCAP_{lk}$: Minimum required resource $k$ for a facility at level $l$ (if built).
        \item $t_{ijl}$: Response time from facility $i$ at level $l$ to site $j$. Higher levels may deploy faster assets (e.g., VTOL drones vs standard quadcopters).
        \item $S_j$: Service Level Agreement (maximum allowable response time) for site $j$.
    \end{itemize}
    \item \textbf{Demand:}
    \begin{itemize}
        \item $D_j$: Total security demand (SCU) at site $j$.
        \item $\alpha_j$: Site-specific mix ratio required to cover demand (Human/Robot balance).
        \item $\alpha$: Global supervision ratio (minimum Humans per Robot).
    \end{itemize}
\end{itemize}

\subsubsection{Decision Variables}
\begin{itemize}
    \item $x_{il} \in \{0, 1\}$: 1 if candidate location $i$ is developed at level $l$, 0 otherwise.
    \item $y_{ij} \in \{0, 1\}$: 1 if demand site $j$ is assigned to facility $i$, 0 otherwise.
    \item $z_{ik} \in \mathbb{Z}^+$: Number of resources of type $k$ deployed at facility $i$.
\end{itemize}

\subsubsection{Objective Function}
Minimize the total monthly cost ($Z$), comprising fixed facility costs and variable resource deployment costs:
\begin{equation}
\text{Minimize } Z = \sum_{i \in I} \sum_{l \in L} F_{il} x_{il} + \sum_{i \in I} \sum_{k \in K} C_{ik} z_{ik}
\end{equation}

\subsubsection{Constraints}
\begin{enumerate}
    \item \textbf{Single Level Selection:} Each candidate location can strictly accommodate at most one facility type.
    \begin{equation}
    \sum_{l \in L} x_{il} \le 1, \quad \forall i \in I
    \end{equation}
    
    \item \textbf{Demand Assignment:} Every demand site must be assigned to exactly one facility.
    \begin{equation}
    \sum_{i \in I} y_{ij} \ge 1, \quad \forall j \in J
    \end{equation}
    
    \item \textbf{Logical Link:} Assignments can only be made to active facilities.
    \begin{equation}
    y_{ij} \le \sum_{l \in L} x_{il}, \quad \forall i \in I, j \in J
    \end{equation}
    
    \item \textbf{SLA Compliance:} A facility at level $l$ can only serve site $j$ if the response time is within the limit.
    \begin{equation}
    t_{ijl} x_{il} \le S_j + M(1 - y_{ij}), \quad \forall i, j, l
    \end{equation}
    
    \item \textbf{Resource Capacity (Upper Bound):} Total resources must not exceed the physical capacity of the chosen level.
    \begin{equation}
    z_{ik} \le \sum_{l \in L} MAXCAP_{lk} x_{il}, \quad \forall i \in I, k \in K
    \end{equation}
    
    \item \textbf{Minimum Resource Requirement (Lower Bound):} Opened facilities must maintain a base level of readiness.
    \begin{equation}
    z_{ik} \ge \sum_{l \in L} MINCAP_{lk} x_{il}, \quad \forall i \in I, k \in K
    \end{equation}
    
    \item \textbf{Coverage Satisfaction:} Resources must be sufficient to cover the split demand (Robotic vs. Human tasks) of all assigned sites.
    \begin{align}
    z_{i, \text{Robot}} &\ge \sum_{j \in J} \frac{D_j}{1+\alpha_j} y_{ij}, \quad \forall i \in I \\
    z_{i, \text{Human}} &\ge \sum_{j \in J} \frac{D_j \alpha_j}{1+\alpha_j} y_{ij}, \quad \forall i \in I
    \end{align}
    
    \item \textbf{Global Supervision:} A global policy constraint ensuring adequate human oversight for robotic fleets.
    \begin{equation}
    z_{i, \text{Human}} \ge \alpha \cdot z_{i, \text{Robot}}, \quad \forall i \in I
    \end{equation}
\end{enumerate}

\section{Method}
The problem is NP-hard, combining facility location and resource allocation decisions. We employ a hybrid computational strategy: an Exact Solver for small instances/validation and a specialized Heuristic for large-scale application.

\subsection{Exact Solution Approach}
We implemented the Mixed Integer Programming (MIP) model using the Gurobi Optimizer. This approach guarantees global optimality by exploring the branch-and-bound tree. It serves as the "ground truth" for validating the heuristic's performance and correctness on tractable instances.

\subsection{Heuristic Algorithm}
For larger instances, we developed a two-stage metaheuristic tailored to the multi-level nature of the problem.

\subsubsection{Stage 1: Multi-Level Constructive Greedy}
The algorithm builds an initial feasible solution by iterating through demand sites and assigning them to the "best" available facility.
\begin{itemize}
    \item \textbf{Level Optimization:} When considering a facility $i$ for a site $j$, the algorithm dynamically calculates the minimum required facility level ($L_{min}$) that satisfies both the response time SLA ($t_{ijl} \le S_j$) and the aggregated capacity capability.
    \item \textbf{Marginal Cost Selection:} The assignment is made to the facility-level pair that minimizes the immediate marginal cost increase (including potential level upgrades or new facility openings).
\end{itemize}

\subsubsection{Stage 2: Local Search Improvement}
We apply a "Best Improvement" local search that iteratively explores the neighborhood of the current solution until no better solution is found. The moves include:
\begin{itemize}
    \item \textbf{Shift Move:} Reassigns a demand site from its current facility to another, potentially adjusting the levels of both facilities to maintain feasibility and optimality.
    \item \textbf{Swap Move:} Exchanges the assignment of two logical sites between facilities.
    \item \textbf{Drop Move:} Attempts to close a facility by redistributing all its assigned demand sites to neighboring facilities, checking if the savings in fixed costs outweigh increased transport/variable costs.
    \item \textbf{Open Move:} Identifies unserved or inefficiently served clusters and attempts to open a new facility (optimizing its level) to reduce total system cost.
\end{itemize}

\section{Data Collection and Generation}
The study utilizes a hybrid dataset combining real-world geospatial topology with parameter synthesis based on industrial security standards.

\subsection{Geospatial Data}
The experimental environment simulates the Dhahran Core Area.
\begin{itemize}
    \item \textbf{Locations:} 15 Candidate locations ($I$) and 50 Demand sites ($J$).
    \item \textbf{Corridor Pattern:} Sites are distributed to mimic pipeline corridors (linear clusters) and scattered high-value assets, reflecting typical petro-infrastructure layouts.
    \item \textbf{Distances:} Geodesic distances ($d_{ij}$) are calculated to realism.
\end{itemize}

\subsection{Facility Levels}
Three command center levels were modeled:
\begin{itemize}
    \item \textbf{High Level:} Premium infrastructure. High fixed cost (1.5x base), but capable of supporting large fleets ($MAXCAP \approx 240$ robots) and equipped with high-speed response units (0.5x travel time).
    \item \textbf{Medium Level:} Standard infrastructure. Base cost and time factors.
    \item \textbf{Low Level:} Satellite outposts. Low fixed cost (0.5x base), limited capacity, and slower response times (1.5x base).
\end{itemize}

\subsection{Scenarios}
To evaluate the impact of technological maturity, we defined three scenarios varying the Global Supervision Ratio ($\alpha$) and Robot Efficiency/Cost:
\begin{itemize}
    \item \textbf{Conservative:} High human reliance ($\alpha \approx 1:3$), standard robot costs.
    \item \textbf{Balanced:} Current technology mix ($\alpha \approx 1:5$), reduced robot costs.
    \item \textbf{Future:} Autonomous-centric ($\alpha \approx 1:10$), low robot costs and high reliance on automation.
\end{itemize}

\section{Results}
This section summarizes the performance of the optimization model across the defined scenarios. The results highlight the substantial economic verification of the proposed multi-level facility model.

\subsection{Optimization Results}
Table~\ref{tab:results} presents the optimal costs and network configurations found by the exact solver.

\begin{table}[ht]
\centering
\begin{tabular}{|l|r|r|r|r|}
\hline
\textbf{Scenario} & \textbf{Total Cost (\$)} & \textbf{Facilities} & \textbf{Solve Time (s)} & \textbf{Cost Reduction} \\ \hline
Conservative & 1,399,600 & 9 & 1.25 & Baseline \\ \hline
Balanced & 879,600 & 7 & 0.21 & 37.1\% \\ \hline
Future & 512,400 & 5 & 3.18 & 63.4\% \\ \hline
\end{tabular}
\caption{Exact Solution Results by Technology Scenario}
\label{tab:results}
\end{table}

The model dynamically selects facility levels to match demand density. In the Conservative scenario, a larger number of facilities (9) are deployed, likely including simpler Low-level outposts to meet the high human density requirement locally. In the Future scenario, the network consolidates into 5 high-efficiency hubs.

\subsection{Heuristic Performance}
The multi-stage heuristic demonstrated robust performance, achieving less than 2.5\% gap from optimality while providing significant speedups.

\begin{table}[ht]
\centering
\begin{tabular}{|l|r|r|r|r|}
\hline
\textbf{Scenario} & \textbf{Exact Cost (\$)} & \textbf{Heuristic Cost (\$)} & \textbf{Gap (\%)} & \textbf{Heuristic Time (s)} \\ \hline
Conservative & 1,399,600 & 1,433,800 & 2.44 & 0.25 \\ \hline
Balanced & 879,600 & 889,600 & 1.14 & 0.29 \\ \hline
Future & 512,400 & 517,400 & 0.98 & 0.12 \\ \hline
\end{tabular}
\caption{Exact vs. Heuristic Comparison}
\label{tab:heuristic}
\end{table}

\subsection{Resource Allocation}
The supervision constraints significantly influence the optimal mix.

\begin{table}[ht]
\centering
\begin{tabular}{|l|c|c|c|c|}
\hline
\textbf{Scenario} & \textbf{$\alpha$ (H:R Ratio)} & \textbf{Robot Eff.} & \textbf{Est. Robots} & \textbf{Est. Humans} \\ \hline
Conservative & 1:3 & 1.5x & 240 & 80 \\ \hline
Balanced & 1:5 & 3.0x & 180 & 36 \\ \hline
Future & 1:10 & 5.0x & 120 & 12 \\ \hline
\end{tabular}
\caption{Estimated Resource Allocation}
\label{tab:resources}
\end{table}

\subsection{Visualizations}
The following figures illustrate the spatial distribution of the optimized solutions. The exact locations of the opened facilities and their assignments demonstrate the model's ability to balance coverage requirements against infrastructure costs.

\begin{figure}[ht]
\centering
\includegraphics[width=0.85\textwidth]{figures/result_conservative_scenario_-_exact_method.pdf}
\caption{Conservative Scenario Breakdown}
\label{fig:conservative}
\end{figure}

\begin{figure}[ht]
\centering
\includegraphics[width=0.85\textwidth]{figures/result_balanced_scenario_-_exact_method.pdf}
\caption{Balanced Scenario Breakdown}
\label{fig:balanced}
\end{figure}

\begin{figure}[ht]
\centering
\includegraphics[width=0.85\textwidth]{figures/result_future_scenario_-_exact_method.pdf}
\caption{Future Scenario Breakdown}
\label{fig:future}
\end{figure}

\subsection{Executable Deployment Plan}
Based on the Balanced (Current Technology) scenario, we present an executable deployment plan for the Dhahran district:

\textbf{Phase 1 - Infrastructure (Months 1-6):}
\begin{enumerate}
    \item Construct 7 command centers at optimal locations identified by the model.
    \item Prioritize locations 8-14 (peripheral sites) which serve the majority of demand.
    \item Total construction investment: \$140,000 (7 × \$20,000 average).
\end{enumerate}

\textbf{Phase 2 - Resource Deployment (Months 7-12):}
\begin{enumerate}
    \item Deploy robotic units (drones/UGVs) at each command center.
    \item Hire and train human supervisors at 1:5 ratio.
    \item Monthly operational cost: \$739,600 (\$879,600 - \$140,000 fixed).
\end{enumerate}

\textbf{Phase 3 - Technology Migration (Years 2-5):}
\begin{enumerate}
    \item Gradually transition toward Future scenario as AI maturity improves.
    \item Consolidate from 7 to 5 facilities, closing underutilized centers.
    \item Target monthly savings: \$367,200 (Balanced to Future difference).
\end{enumerate}

\subsection{Economic Impact Analysis}
Compared to a hypothetical baseline of full human deployment (without robotic augmentation), the proposed solutions offer substantial savings:

\begin{itemize}
    \item \textbf{Conservative Scenario:} Estimated 25\% savings vs. all-human baseline.
    \item \textbf{Balanced Scenario:} Estimated 45\% savings vs. all-human baseline.
    \item \textbf{Future Scenario:} Estimated 65\% savings vs. all-human baseline.
\end{itemize}

These projections assume a fully human-staffed baseline monthly cost of approximately \$1,800,000 (based on Tier 1 security personnel costs in the Eastern Province).

\section{Conclusions}
This study formulated and solved a Capacitated Facility Location Problem tailored for security command center optimization at Saudi Aramco's Dhahran headquarters. The mathematical model successfully integrates multiple real-world constraints including SLA compliance, human-robot supervision ratios, facility capacity limits, and minimum utilization requirements.

\subsection{Key Contributions}
\begin{enumerate}
    \item \textbf{Novel Problem Formulation:} We adapted the classical CFLP framework to explicitly model the trade-off between human security personnel and autonomous robotic units, incorporating supervision constraints unique to industrial security applications.
    
    \item \textbf{Computational Framework:} We developed a hybrid solution approach combining exact optimization (Gurobi) for benchmark solutions with a constructive greedy heuristic enhanced by Shift, Swap, and Drop/Open local search moves. The heuristic achieves less than 2.5\% optimality gap while running 10x faster.
    
    \item \textbf{Scenario Analysis:} By varying technological maturity parameters while keeping geography constant, we quantified the potential cost savings from automation: up to 63.4\% reduction when transitioning from conservative to fully autonomous operations.
    
    \item \textbf{Deployment Roadmap:} The executable deployment plan provides Saudi Aramco with a phased approach to implement the optimal solution, starting with current technology and evolving toward higher automation as AI capabilities mature.
\end{enumerate}

\subsection{Limitations}
\begin{itemize}
    \item \textbf{Synthetic Data:} While coordinates are based on the real Dhahran topology, specific demand values and cost parameters were synthetically generated based on industry estimates rather than actual Aramco operational data.
    \item \textbf{Static Model:} The current formulation assumes static demand; future work could incorporate time-varying demand patterns (day/night shifts, seasonal variations).
    \item \textbf{Single-Period Planning:} The model optimizes a single planning period; multi-period capacity expansion models could capture facility construction sequencing more accurately.
\end{itemize}

\subsection{Future Research Directions}
\begin{enumerate}
    \item \textbf{Stochastic Demand:} Incorporate uncertainty in security demand using robust or chance-constrained optimization.
    \item \textbf{Dynamic Response Modeling:} Add explicit travel time modeling with routing constraints rather than simple SLA distance limits.
    \item \textbf{Multi-Objective Optimization:} Balance cost minimization against response time minimization and coverage maximization using Pareto-optimal frontiers.
    \item \textbf{Real-World Validation:} Collaborate with Saudi Aramco to validate the model using actual operational data and conduct pilot deployments.
    \item \textbf{Integration with IoT:} Extend the framework to incorporate real-time sensor data for adaptive resource reallocation.
\end{enumerate}

\subsection{Final Remarks}
The optimization framework developed in this study demonstrates that significant operational cost savings are achievable through strategic facility location and human-robot resource allocation. As autonomous security technologies continue to mature, organizations like Saudi Aramco can leverage such decision-support tools to systematically plan their transition toward Industry 4.0 security operations while maintaining the human oversight essential for ethical and effective decision-making. Source code available at: \url{https://github.com/naimackerman/aramco_security_opt}


% ---------- Highlighting example ----------
\subsubsection{Highlighting Text or Citations}
\begin{highlight}[0.5in]
Text of this category must be italicized, justified, and have .5-inch margins all around. This ensures that important material is highlighted facilitating meaning conveyance.
\end{highlight}

\subsubsection{Recommendations}
We strongly encourage you to use this document as a template for developing your own manuscript.

% ---------- References (actual reference list) ----------
% ---- Format references as shown below. Citations and references must comply with the APA reference style. For the initial paper submission, do not include title or author information in references to previous work by the paper’s author. Include full reference information for the final paper submission.

% ---- Note 1: Begin the Refence section on a new page
% ---- Note 2: If you have a reference manager, use APA 7th.
% ---- Note 3: For the initial paper submission, do not include title or author information in references to previous work by the paper’s author. Include full reference information for the final paper submission

\newpage
\nocite{*}
\section{References}
\printbibliography[heading=none]

% ---------- Biography Format ----------
\newpage
\phantomsection
\makeatletter
\renewcommand{\authorbioentry}[3]{%
  \noindent\begin{tabular}{@{}m{0.5in} M{\dimexpr\columnwidth-0.5in\relax}@{}}
    \authorpic{#1} &
    {\headingfont\bfseries\raggedright\fontsize{12pt}{14pt}\selectfont #2}\par #3
  \end{tabular}\par\medskip
}
\makeatother
% ---------- Biography ----------
% ---- Note : Begin the Biography section on a new page

\section*{Biography}
\authorbioentry{template-images/author1_pic.jpg}{Author Name}{Provide a short biography of the author. Provide a short biography of the author.}
\authorbioentry{template-images/author2_pic.jpg}{Second Author}{Provide a short biography of the second author.}
\authorbioentry{template-images/author3_pic.jpg}{Third Author}{Provide a short biography of the third author.}

\end{multicols*}

\end{document}